\section{Introdução} \label{sec:Intro}
A gestão eficiente de resíduos sólidos é um desafio crescente nas sociedades modernas. A busca por soluções inovadoras que tornem esse processo mais prático e sustentável tem conduzido a avanços tecnológicos em várias áreas.\cite{sensor}\footnote{Referêcia detalhada sobre sensor utilizado.} e com o objetivo de desenvolver
de uma lixeira automática equipada com sensores  e mecanismos de automação que visem otimizar a coleta e a limpeza de lixeiras públicas este trabalho foi desenvolvido.

A proposta se baseia em um sistema composto por uma série de etapas coordenadas, com o objetivo de melhorar a eficiência da coleta de resíduos, garantindo que a lixeira esteja sempre pronta para receber novos resíduos de maneira higiênica e eficaz. Cada etapa do processo é cuidadosamente planejada e automatizada, permitindo que a lixeira realize a limpeza e a preparação para a próxima coleta de maneira automatica.

A lixeira automática traz consigo um novo meio de explorar a sustentabilidade das grandes cidades e da sua interação com o meio ambiente, no entanto surge a necessidade de implementar na sua automação aparelhos que tornem essa automatização possível, sendo esses componentes, conjuntos de aparelhos que trabalham de maneira uniforme e que atingem o esperado, sendo a limpeza e a lacração automática da sacola de lixo. Dito isso, a Lixeira Automática é equipada com sensores avançados, atuadores e um sistema de controle baseado na plataforma Raspberry Pi \cite{raspberry}\footnote{Referência detalhada sobre Raspberry Pi.}, tornando-a capaz de detectar o nível de enchimento da lixeira, a presença de uma bolsa de resíduos, e realizar uma sequência de ações automáticas, incluindo a limpeza da lixeira e o selamento da sacola, tornando-a pronta para o próximo ciclo de coleta.

Este trabalho aborda detalhadamente o projeto, implementação e avaliação da Lixeira Automática. Ele descreve o \textit{hardware} e \textit{software} envolvidos, a lógica de automação, a calibração de sensores, a eficiência energética, os testes práticos, a sustentabilidade e o impacto social do sistema. Além disso, compara a lixeira automática com as soluções tradicionais de coleta de resíduos em termos de eficiência, custo e sustentabilidade.

O objetivo deste projeto é proporcionar uma contribuição significativa para a gestão de resíduos urbanos, criando um sistema inovador que promova a limpeza, a sustentabilidade e a qualidade de vida nas áreas urbanas. Ao explorar as potencialidades da tecnologia, este trabalho visa inspirar futuras soluções criativas e eficientes para os desafios urbanos contemporâneos de gestão de resíduos. Na próxima seção, apresentamos os objetivos gerais e específicos desta proposta.


