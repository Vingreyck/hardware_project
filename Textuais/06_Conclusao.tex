\section{Conclusões}

Este trabalho apresentou a \textit{Lixeira Inteligente}, uma proposta que representa uma solução inovadora e promissora para a gestão eficiente de resíduos sólidos em ambientes urbanos. Ao longo deste trabalho, foram explorados diversos aspectos, desde o projeto e a implementação do \textit{hardware} e \textit{software} até a avaliação do desempenho e da viabilidade da lixeira inteligente.

O sistema foi projetado com a finalidade de otimizar a coleta, lacração e limpeza de resíduos, tornando o processo mais eficiente, higiênico e sustentável. A integração de sensores, atuadores e automação permite que a lixeira determine o nível de enchimento, execute a limpeza automática e garanta a prontidão para receber novos resíduos.

Uma das características distintivas desse projeto é a ênfase na eficiência energética, com o objetivo de minimizar o consumo de recursos e reduzir o impacto ambiental. O uso de sensores ultrassônicos HC-SR04 proporciona uma medição precisa do nível de enchimento da lixeira, enquanto a automação, controlada pela Raspberry Pi, coordena todo o processo.

Este projeto não é apenas uma resposta inovadora ao desafio da gestão de resíduos, mas também um exemplo do potencial da tecnologia para aprimorar a vida nas cidades. À medida que a urbanização continua a crescer, soluções inteligentes e sustentáveis são essenciais para enfrentar os desafios associados à expansão urbana e à preservação do meio ambiente. A lixeira automática representa um passo significativo em direção a um futuro mais limpo, eficiente e sustentável, onde a tecnologia desempenha um papel crucial na melhoria das condições de vida nas áreas urbanas e na promoção da responsabilidade ambiental. O projeto não foi concretizado mas este artigo ajudará neste procedimento, com o intuito de promover a sua concretização.

